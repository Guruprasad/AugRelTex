There are three major display techniques for Augmented Reality are head–mounted displays, handheld displays, and spatial displays.

\subsection{Head–mounted}

A Head Mounted Display (HMD) places images of both the physical world and registered virtual graphical objects over the user's view of the world. The HMD's are either optical see–through or video see–through. Optical see-through employs half-silver mirrors to pass images through the lens and overlay information to be reflected into the user's eyes. The HMD must be tracked with sensor that provides six degrees of freedom. This tracking allows the system to align virtual information to the physical world. The main advantage of HMD AR is the user's immersive experience. The graphical information is slaved to the view of the user.[14]

\subsection{Handheld} 

Handheld displays employ a small display that fits in a user's hand. All handheld AR solutions to date opt for video see-through. Initially handheld AR employed fiduciary markers, and later GPS units and MEMS sensors such as digital compasses and six degrees of freedom accelerometer–gyroscope. Today SLAM markerless trackers such as PTAM are starting to come into use. Handheld display AR promises to be the first commercial success for AR technologies. The two main advantages of handheld AR is the portable nature of handheld devices and ubiquitous nature of camera phones. The disadvantages are the physical constraints of the user having to hold the handheld device out in front of them at all times as well as distorting effect of classically wide-angled mobile phone cameras when compared to the real world as viewed through the eye. [15]

\subsection{Spatial}

Instead of the user wearing or carrying the display such as with head mounted displays or handheld devices; Spatial Augmented Reality (SAR) makes use of digital projectors to display graphical information onto physical objects. The key difference in SAR is that the display is separated from the users of the system. Because the displays are not associated with each user, SAR scales naturally up to groups of users, thus allowing for collocated collaboration between users. SAR has several advantages over traditional head mounted displays and handheld devices. The user is not required to carry equipment or wear the display over their eyes. This makes spatial AR a good candidate for collaborative work, as the users can see each other’s faces. A system can be used by multiple people at the same time without each having to wear a head mounted display. Spatial AR does not suffer from the limited display resolution of current head mounted displays and portable devices. A projector based display system can simply incorporate more projectors to expand the display area. Where portable devices have a small window into the world for drawing, a SAR system can display on any number of surfaces of an indoor setting at once. The drawbacks, however, are that SAR systems of projectors do not work so well in sunlight and also require a surface on which to project the computer-generated graphics. Augmentations cannot simply hang in the air as they do with handheld and HMD-based AR. The tangible nature of SAR, though, makes this an ideal technology to support design, as SAR supports both a graphical visualisation and passive haptic sensation for the end users. People are able to touch physical objects, and it is this process that provides the passive haptic sensation. [2] [11] [16] [17][18]
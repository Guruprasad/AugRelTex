Virtual reality is a technology that encompasses a broad spectrum of ideas. It defines an umbrella under which many researchers and companies express their work. The phrase was originated by Jaron Lanier the founder of VPL Research one of the original companies selling virtual reality systems. The term was defined as "a computer generated, interactive, three-dimensional environment in which a person is immersed." (Aukstakalnis and Blatner 1992) There are three key points in this definition. First, this virtual environment is a computer generated three-dimensional scene which requires high performance computer graphics to provide an adequate level of realism. The second point is that the virtual world is interactive. A user requires real-time response from the system to be able to interact with it in an effective manner. The last point is that the user is immersed in this virtual environment. One of the identifying marks of a virtual reality system is the head mounted display worn by users. These displays block out all the external world and present to the wearer a view that is under the complete control of the computer. The user is completely immersed in an artificial world and becomes divorced from the real environment. For this immersion to appear realistic the virtual reality system must accurately sense how the user is moving and determine what effect that will have on the scene being rendered in the head mounted display.

The discussion above highlights the similarities and differences between virtual reality and augmented reality systems. A very visible difference between these two types of systems is the immersiveness of the system. Virtual reality strives for a totally immersive environment. The visual, and in some systems aural and proprioceptive, senses are under control of the system. In contrast, an augmented reality system is augmenting the real world scene necessitating that the user maintains a sense of presence in that world. The virtual images are merged with the real view to create the augmented display. There must be a mechanism to combine the real and virtual that is not present in other virtual reality work. Developing the technology for merging the real and virtual image streams is an active research topic and is briefly described in Section 1.3.3.

The computer generated virtual objects must be accurately registered with the real world in all dimensions. Errors in this registration will prevent the user from seeing the real and virtual images as fused. The correct registration must also be maintained while the user moves about within the real environment. Discrepancies or changes in the apparent registration will range from distracting which makes working with the augmented view more difficult, to physically disturbing for the user making the system completely unusable. An immersive virtual reality system must maintain registration so that changes in the rendered scene match with the perceptions of the user. Any errors here are conflicts between the visual system and the kinesthetic or proprioceptive systems. The phenomenon of visual capture gives the vision system a stronger influence in our perception (Welch 1978). This will allow a user to accept or adjust to a visual stimulus overriding the discrepancies with input from sensory systems. In contrast, errors of misregistration in an augmented reality system are between two visual stimuli which we are trying to fuse to see as one scene. We are more sensitive to these errors (Azuma 1993; Azuma 1995).

Milgram (Milgram and Kishino 1994; Milgram, Takemura et al. 1994) describes a taxonomy that identifies how augmented reality and virtual reality work are related. He defines the Reality-Virtuality continuum shown as Figure 2.

%Figure 2 - Milgram's Reality-Virtuality Continuum

The real world and a totally virtual environment are at the two ends of this continuum with the middle region called Mixed Reality. Augmented reality lies near the real world end of the line with the predominate perception being the real world augmented by computer generated data. Augmented virtuality is a term created by Milgram to identify systems which are mostly synthetic with some real world imagery added such as texture mapping video onto virtual objects. This is a distinction that will fade as the technology improves and the virtual elements in the scene become less distinguishable from the real ones.

Milgram further defines a taxonomy for the Mixed Reality displays. The three axes he suggests for categorizing these systems are: Reproduction Fidelity, Extent of Presence Metaphor and Extent of World Knowledge. Reproduction Fidelity relates to the quality of the computer generated imagery ranging from simple wireframe approximations to complete photorealistic renderings. The real-time constraint on augmented reality systems forces them to be toward the low end on the Reproduction Fidelity spectrum. The current graphics hardware capabilities can not produce real-time photorealistic renderings of the virtual scene. Milgram also places augmented reality systems on the low end of the Extent of Presence Metaphor. This axis measures the level of immersion of the user within the displayed scene. This categorization is closely related to the display technology used by the system. There are several classes of displays used in augmented reality systems that are discussed in Section 1.3.3. Each of these gives a different sense of immersion in the display. In an augmented reality system, this can be misleading because with some display technologies part of the "display" is the user's direct view of the real world. Immersion in that display comes from simply having your eyes open. It is contrasted to systems where the merged view is presented to the user on a separate monitor for what is sometimes called a "Window on the World" view.

The third, and final, dimension that Milgram uses to categorize Mixed Reality displays is Extent of World Knowledge. Augmented reality does not simply mean the superimposition of a graphic object over a real world scene. This is technically an easy task. One difficulty in augmenting reality, as defined here, is the need to maintain accurate registration of the virtual objects with the real world image. As will be described in Section 1.3.5, this often requires detailed knowledge of the relationship between the frames of reference for the real world, the camera viewing it and the user. In some domains these relationships are well known which makes the task of augmenting reality easier or might lead the system designer to use a completely virtual environment. The contribution of this thesis will be to minimize the calibration and world knowledge necessary to create an augmented view of the real environment.
\subsection{Entertainment}

A simple form of augmented reality has been in use in the entertainment and news business for quite some time. Whenever you are watching the evening weather report the weather reporter is shown standing in front of changing weather maps. In the studio the reporter is actually standing in front of a blue or green screen. This real image is augmented with computer generated maps using a technique called chroma-keying. It is also possible to create a virtual studio environment so that the actors can appear to be positioned in a studio with computer generated decorating. Examples of using this technique can be found at (Schmidt 1996; Schmidt 1996b).

Movie special effects make use of digital compositing to create illusions (Pyros and Goren 1995). Strictly speaking with current technology this may not be considered augmented reality because it is not generated in real-time. Most special effects are created off-line, frame by frame with a substantial amount of user interaction and computer graphics system rendering. But some work is progressing in computer analysis of the live action images to determine the camera parameters and use this to drive the generation of the virtual graphics objects to be merged (Zorpette 1994).

Princeton Electronic Billboard has developed an augmented reality system that allows broadcasters to insert advertisements into specific areas of the broadcast image (National Association of Broadcasters 1994). For example, while broadcasting a baseball game this system would be able to place an advertisement in the image so that it appears on the outfield wall of the stadium. The electronic billboard requires calibration to the stadium by taking images from typical camera angles and zoom settings in order to build a map of the stadium including the locations in the images where advertisements will be inserted. By using pre-specified reference points in the stadium, the system automatically determines the camera angle being used and referring to the pre-defined stadium map inserts the advertisement into the correct place. The approach used for mapping these planar surfaces is similar to that used by the U of R augmented reality system described in Section 2.

\subsection{Military Training}

The military has been using displays in cockpits that present information to the pilot on the windshield of the cockpit or the visor of their flight helmet. This is a form of augmented reality display. SIMNET, a distributed war games simulation system, is also embracing augmented reality technology. By equipping military personnel with helmet mounted visor displays or a special purpose rangefinder (Urban 1995) the activities of other units participating in the exercise can be imaged. While looking at the horizon, for example, the display equipped soldier could see a helicopter rising above the tree line (Metzger 1993). This helicopter could be being flown in simulation by another participant. In wartime, the display of the real battlefield scene could be augmented with annotation information or highlighting to emphasize hidden enemy units.

\subsection{Engineering Design}

Imagine that a group of designers are working on the model of a complex device for their clients. The designers and clients want to do a joint design review even though they are physically separated. If each of them had a conference room that was equipped with an augmented reality display this could be accomplished. The physical prototype that the designers have mocked up is imaged and displayed in the client's conference room in 3D. The clients can walk around the display looking at different aspects of it. To hold discussions the client can point at the prototype to highlight sections and this will be reflected on the real model in the augmented display that the designers are using. Or perhaps in an earlier stage of the design, before a prototype is built, the view in each conference room is augmented with a computer generated image of the current design built from the CAD files describing it. This would allow real time interaction with elements of the design so that either side can make adjustments and changes that are reflected in the view seen by both groups (Ahlers, Kramer et al. 1995). A technique for interactively obtaining a model for 3D objects called 3D stenciling that takes advantage of an augmented reality display is being investigated in our department by Kyros Kutulakos.

\subsection{Robotics and Telerobotics}

In the domain of robotics and telerobotics an augmented display can assist the user of the system (Kim, Schenker et al. 1993; Milgram, Zhai et al. 1993). A telerobotic operator uses a visual image of the remote workspace to guide the robot. Annotation of the view would still be useful just as it is when the scene is in front of the operator. There is an added potential benefit. Since often the view of the remote scene is monoscopic, augmentation with wireframe drawings of structures in the view can facilitate visualization of the remote 3D geometry. If the operator is attempting a motion it could be practiced on a virtual robot that is visualized as an augmentation to the real scene. The operator can decide to proceed with the motion after seeing the results. The robot motion could then be executed directly which in a telerobotics application would eliminate any oscillations caused by long delays to the remote site. More information about augmented reality in robotics can be found at (Milgram 1995).

\subsection{Manufacturing, Maintenance and Repair}

When the maintenance technician approaches a new or unfamiliar piece of equipment instead of opening several repair manuals they could put on an augmented reality display. In this display the image of the equipment would be augmented with annotations and information pertinent to the repair. For example, the location of fasteners and attachment hardware that must be removed would be highlighted. Then the inside view of the machine would highlight the boards that need to be replaced (Feiner, MacIntyre et al. 1993; Uenohara and Kanade 1995). An example of augmented reality being used for maintenance can be seen at (Feiner 1995). The military has developed a wireless vest worn by personnel that is attached to an optical see-through display (Urban 1995). The wireless connection allows the soldier to access repair manuals and images of the equipment. Future versions might register those images on the live scene and provide animation to show the procedures that must be performed.

Boeing researchers are developing an augmented reality display to replace the large work frames used for making wiring harnesses for their aircraft (Caudell 1994; Sims 1994). Using this experimental system, the technicians are guided by the augmented display that shows the routing of the cables on a generic frame used for all harnesses. The augmented display allows a single fixture to be used for making the multiple harnesses.

\subsection{Consumer Design}

Virtual reality systems are already used for consumer design. Using perhaps more of a graphics system than virtual reality, when you go to the typical home store wanting to add a new deck to your house, they will show you a graphical picture of what the deck will look like. It is conceivable that a future system would allow you to bring a video tape of your house shot from various viewpoints in your backyard and in real time it would augment that view to show the new deck in its finished form attached to your house. Or bring in a tape of your current kitchen and the augmented reality processor would replace your current kitchen cabinetry with virtual images of the new kitchen that you are designing.

Applications in the fashion and beauty industry that would benefit from an augmented reality system can also be imagined. If the dress store does not have a particular style dress in your size an appropriate sized dress could be used to augment the image of you. As you looked in the three sided mirror you would see the image of the new dress on your body. Changes in hem length, shoulder styles or other particulars of the design could be viewed on you before you place the order. When you head into some high-tech beauty shops today you can see what a new hair style would look like on a digitized image of yourself. But with an advanced augmented reality system you would be able to see the view as you moved. If the dynamics of hair are included in the description of the virtual object you would also see the motion of your hair as your head moved.
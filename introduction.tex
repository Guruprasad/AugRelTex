Augmented reality (AR) \cite{wikiAR} is a term for a live direct or an indirect view of a physical, real-world environment whose elements are augmented by computer-generated sensory input, such as sound or graphics. It is related to a more general concept called mediated reality, in which a view of reality is modified (possibly even diminished rather than augmented) by a computer. As a result, the technology functions by enhancing one’s current perception of reality. By contrast, virtual reality replaces the real-world with a simulated one.

Augmentation is conventionally in real-time and in semantic context with environmental elements, such as sports scores on TV during a match. With the help of advanced AR technology (e.g. adding computer vision and object recognition) the information about the surrounding real world of the user becomes interactive and digitally manipulable. Artificial information about the environment and its objects can be overlaid on the real world. The term augmented reality is believed to have been coined in 1990 by Thomas Caudell, working at Boeing.

Research explores the application of computer-generated imagery in live-video streams as a way to enhance the perception of the real world. AR technology includes head-mounted displays and virtual retinal displays for visualization purposes, and construction of controlled environments containing sensors and actuators. 
\label{ http://en.wikipedia.org/wiki/Augmented_reality}

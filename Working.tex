Augmented Reality techniques are not new, they employ already known
technologies to accomplish their goal. In a standard setup, the system
consists of an input source like a mobile phone camera or a head mounted 
camera unit , a processing unit, and a display unit. The input is the
basis on which the information is augmented.

To describe the working , let us take a simple example of projecting
a 3D object on a sheet of paper.
There are a few key requirements:
\begin{enumerate}
\item 
The object must be projected at the exact place identified for it.
\item
The object must react to changes in the real world.(If the paper
is moved the object must move as well.)
\end{enumerate}

\vspace*{5px}
The Following steps are followed:
\begin{enumerate}

\item
We place the real object (a sheet of paper, with a marked
area , a square box, where the 3D object must be projected) in front
of the camera.



\item
Our object recognition algorithm then identifies that there
is a marked area on the paper , and its aware that it is a square.

\item
The algorithm then creates the projection of the square in 3D space.

\item
The algorithm transforms the the 3D object (a cuboid) to match the
projection of the square.

\item
The algorithm redraws every frame coming from the camera with the
3D object drawn above the square.And displays it using.


\item
Now if the user moves the piece of paper, the orientation of the
square in 3D space changes , then steps 3,4,5 are repeated.


\end{enumerate}

%\end{enumerate}
